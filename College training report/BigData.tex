Big data is a term that describes the large volume of data – both structured and unstructured – that inundates a business on a day-to-day basis. But it’s not the amount of data that’s important. It’s what organizations do with the data that matters. Big data can be analyzed for insights that lead to better decisions and strategic business moves.

\section{Big Data Characteristics}
Big data can be described by the following characteristics:
\begin{enumerate}
\item Volume
The quantity of generated and stored data. The size of the data determines the value and potential insight- and whether it can actually be considered big data or not.
\item Variety
The type and nature of the data. This helps people who analyze it to effectively use the resulting insight.
\item Velocity
In this context, the speed at which the data is generated and processed to meet the demands and challenges that lie in the path of growth and development.
\item Variability
Inconsistency of the data set can hamper processes to handle and manage it.
\item Complexity
Today's data comes from multiple sources, which makes it difficult to link, match, cleanse and transform data across systems. However, it’s necessary to connect and correlate relationships, hierarchies and multiple data linkages or your data can quickly spiral out of control.
\end{enumerate}

\section{Importance of Big Data}
The importance of big data doesn’t revolve around how much data you have, but what you do with it.
You can take data from any source and analyze it to find answers that enable
\begin{itemize}

\item Cost reductions
\item Time reductions
\item New product development and optimized offerings
\item Smart decision making.
\end{itemize}

When you combine big data with high-powered analytics, you can accomplish business-related tasks such as:
\begin{itemize}
\item Determining root causes of failures, issues and defects in near-real time.
\item Generating coupons at the point of sale based on the customer’s buying habits.
\item Recalculating entire risk portfolios in minutes.
\item Detecting fraudulent behaviour before it affects your organization.
\end{itemize}

\section{Uses of Big Data}
Big data affects organizations across practically every industry. See how each industry can benefit from this onslaught of information.
\begin{itemize}
\item Banking


With large amounts of information streaming in from countless sources, banks are faced with finding new and innovative ways to manage big data. While it’s important to understand customers and boost their satisfaction, it’s equally important to minimize risk and fraud while maintaining regulatory compliance. Big data brings big insights, but it also requires financial institutions to stay one step ahead of the game with advanced analytics.

\item Education


Educators armed with data-driven insight can make a significant impact on school systems, students and curriculums. By analyzing big data, they can identify at-risk students, make sure students are making adequate progress, and can implement a better system for evaluation and support of teachers and principals.

\item Government


When government agencies are able to harness and apply analytics to their big data, they gain significant ground when it comes to managing utilities, running agencies, dealing with traffic congestion or preventing crime. But while there are many advantages to big data, governments must also address issues of transparency and privacy.
\end{itemize}
