Data mining is the computing process of discovering patterns in large data sets involving methods at the intersection of machine learning, statistics, and database systems. It is an interdisciplinary subfield of computer science. The overall goal of the data mining process is to extract information from a data set and transform it into an understandable structure for further use. Aside from the raw analysis step, it involves database and data management aspects, data pre-processing, model and inference considerations, interestingness metrics, complexity considerations, post-processing of discovered structures, visualization, and online updating. Data mining is the analysis step of the "knowledge discovery in databases" process, or KDD.
\section{Data Mining Parameters}
In data mining, association rules are created by analyzing data for frequent if/then patterns, then using the support and confidence criteria to locate the most important relationships within the data. Support is how frequently the items appear in the database, while confidence is the number of times if/then statements are accurate.
Other data mining parameters include Sequence or Path Analysis, Classification, Clustering and Forecasting. Sequence or Path Analysis parameters look for patterns where one event leads to another later event. A Sequence is an ordered list of sets of items, and it is a common type of data structure found in many databases. A Classification parameter looks for new patterns, and might result in a change in the way the data is organized. Classification algorithms predict variables based on other factors within the database.

\section{Benefits Of Data Mining}
In general, the benefits of data mining are -
\begin{itemize}
\item The ability to uncover hidden patterns
\item Uncover relationships in data
\item Can be used to make predictions that impact businesses.
\item Sales and marketing departments can mine customer data.
\end{itemize}
