Psychology is the science of behaviour and mind, embracing all aspects of conscious and unconscious experience as well as thought. It is an academic discipline and a social science which seeks to understand individuals and groups by establishing general principles and researching specific cases.
In this field, a professional practitioner or researcher is called a psychologist and can be classified as a social, behavioural, or cognitive scientist. Psychologists attempt to understand the role of mental functions in individual and social behaviour, while also exploring the physiological and biological processes that underlie cognitive functions and behaviours.
Psychologists explore behaviour and mental processes, including perception, cognition, attention, emotion (affect), intelligence, phenomenology, motivation (conation), brain functioning, and personality. This extends to interaction between people, such as interpersonal relationships, including psychological resilience, family resilience, and other areas. Psychologists of diverse orientations also consider the unconscious mind.
Humanistic psychology is a psychological perspective that emphasizes the study of the whole person. Humanistic psychologists look at human behaviour not only through the eyes of the observer, but through the eyes of the person doing the behaving. Humanistic psychologists believe that an individual's behaviour is connected to his inner feelings and self-image.
Unlike the behaviourists, humanistic psychologists believe that humans are not solely the product of their environment. Rather humanistic psychologists study human meanings, understandings, and experiences involved in growing, teaching, and learning. They emphasize characteristics that are shared by all human beings such as love, grief, caring, and self-worth.
Humanistic psychologists study how people are influenced by their self-perceptions and the personal meanings attached to their experiences. Humanistic psychologists are not primarily concerned with instinctual drives, responses to external stimuli, or past experiences. Rather, they consider conscious choices, responses to internal needs, and current circumstances to be important in shaping human behaviour.
