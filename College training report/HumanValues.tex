Human values have been employed in so distinctively different ways in human discourse. It is
often said that a person has a value or an object has a value. These two usages have been
explicitly recognised by writers in various disciplines such as Charles Mortris in Philosophy,
Brevster Smith in Psychology and Roibin Williams in Sociology.
If one wants to know the origin of the term ‘VALUE’, it may be stated very firmly that the term
‘VALUE’ comes from the Latin word ‘VALERE’ which means ‘to be of worth’. Whereas, the
concise Oxford Dictionary defines the term VALUE’ as the ‘worth, desirability or utility of a
thing’.


In fact, it is difficult to define values, for they are as comprehensive in a nature as our human
life. Somewhere, some other dictionary states that Value is that which renders anything useful,
worthy or estimable. It is price, worth or importance of a thing’.A
Value is “a concept explicit of implicit, distinctive of an individual or characteristics of a group
of those desirable traits which influence the selection from available modes and ends of
action.”


In fact, value is an abstract term which is commonly regarded as an economic conception. In
the words of John Dewey, “Value means primarily, to price, to esteem, to appraise, to estimate.
It means the act of cherishing something holding it clear and also, the act of passing judgement
upon the nature and amount of its value as compared with something else,”
According to Rokeach, “Value is an enduring belief, a specific mode of conduct or an end state
of existence, along a continuum of relative importance.”
Values are part and parcel of philosophy. Hence, aims of education are naturally concerned
with values. Ail education is, in fact, very naturally value-oriented. Each educational goal,
whether originating in a person, a family, a community, a school or an educational system, is
believed to be good. ‘Good’ is intended to mean here ‘avoidance of bad’.
If possible objective, is not good, then there is no reason for pursuing. But again, the same
question spurts out, and when the question ‘what is a value?’ spurts out, we know something of its religion, philosophy and ideology.
The guiding social aims and beliefs which are regarded as the important aspects of a culture,
then, the different aspects of culture are also ‘valued’ by the people; and the ideas lying behind
which they think worthwhile, are called as VALUES!


Values are defined as something which are desirable and worthy of esteem for their own sake.
Human values are defined as those values which help man to live in harmony with the world.
Values that may be included in the general definition of human values are love, brotherhood,
respect for others — including plants and animals — honesty, sincerity, truthfulness, nonviolence, gratitude, tolerance, a sense of responsibility, cooperation, self-reliance, secularism and internationalism.
