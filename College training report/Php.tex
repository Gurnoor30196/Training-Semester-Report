PHP is a server-side scripting language designed primarily for web development but
also used as a general-purpose programming language. Originally created by Rasmus
Lerdorf  in 1994, the PHP reference implementation is now produced by The PHP Development
Team. PHP originally stood for Personal Home Page, but it now stands for the recursive
acronym PHP: Hypertext Pre-processor.
\begin{figure}[ht]
\centering
\includegraphics[scale=0.5]{images/php.png}
\caption{PHP logo}
\end{figure}

PHP code may be embedded into HTML or HTML5 mark-up, or it can be used in combination
with various web template systems, web content management systems and web frameworks.
PHP code is usually processed by a PHP interpreter implemented as a module in the web
server or as a Common Gateway Interface (CGI) executable. The web server software combines
the results of the interpreted and executed PHP code, which may be any type of data,
including images, with the generated web page. PHP code may also be executed with a command-line
interface (CLI) and can be used to implement standalone graphical applications.
The standard PHP interpreter, powered by the Zend Engine, is free software released under the PHP
License. PHP has been widely ported and can be deployed on most web servers on almost every operating system and platform, free of charge.
 The PHP language evolved without a written formal specification or standard until 2014, leaving the canonical PHP interpreter as a de facto standard. Since 2014 work has gone on to create a formal PHP specification.
