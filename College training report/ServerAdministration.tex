Server Administration is an advanced computer networking topic that includes server installation and configuration, server roles, storage, Active Directory and Group Policy, file, print, and web services, remote access, virtualization, application servers, troubleshooting, performance, and reliability.
This course is comprised of 15 lessons that use Windows Server to study and experiment with server administration. Each lesson includes a combination of Wikipedia and Microsoft readings, YouTube videos, and hands-on learning activities. The course also assists learners in preparing for the Microsoft MTA Exam 98-365: Windows Server Administration Fundamentals.
This entire Wikiversity course can be downloaded in book form by selecting Download Learning Guide in the sidebar. The corresponding Wikipedia reading collection can be downloaded in book form by selecting Download Reading Guide.
\section{Server Administrator}
A server administrator or admin has the overall control of a server. This is usually in the context of a business organization, where a server administrator oversees the performance and condition of multiple servers in the business organization, or it can be in the context of a single person running a game server.
The Server Administrator's role is to design, install, administer, and optimize company servers and related components to achieve high performance of the various business functions supported by the servers as necessary. This includes ensuring the availability of client/server applications, configuring all new implementations, and developing processes and procedures for ongoing management of the server environment. Where applicable, the Server Administrator will assist in overseeing the physical security, integrity, and safety of the data centre/server farm.
\section{System Administrator Duties And Skills}
Due to the wide range of job responsibilities for system administrators in various organizations, system administrators' job skill requirements are often broad, as are salary ranges. In general, sysadmins must be comfortable working with application and file servers, desktops, networks, databases, information security systems and storage. Familiarity with multiple operating systems, as well as scripting and programming, is often required. Increasingly, virtualization and cloud computing skills have also become essential to the job.
Because tasks generally include provisioning, configuring and managing physical and virtual servers, as well as the software that runs on the servers and the hardware that supports them, a system administrator should feel comfortable installing and troubleshooting IT resources, establishing and managing user accounts, upgrading and patching software, and performing backup and recovery tasks.
Nontechnical skills are equally important for sysadmins. Because the system administrator interacts with people in so many areas of IT and business, soft skills (people skills) are just as necessary as hard skills. When IT services are slow or down entirely, a system administrator must be able to work under pressure, read a situation as it unfolds and quickly decide upon a response that yields the best result for all involved. 
\section{IT System Administrator Certifications}
System administrators are expected to have at least one, but preferably multiple, certifications for the job. Depending on the technologies used within an enterprise, common certifications in demand include Microsoft Certified Solutions Associate (MCSA), CompTIA Server+, Cisco Certified Network Associate (CCNA) and Red Hat Certified System Administrator (RHCSA).
