\section{Introduction to Project}
\subsection{Overview}
\begin{figure}[ht]
\centering
\includegraphics[scale=0.5]{input/images/paigaam.png}
\caption{Paigaam Logo}
\end{figure}
This Service is the one stop solution for all communication needs for the students and staff/faculty members
inside and outside of the college campus. It has an android based client that communicates with the server
using XMPP protocol (Extensible Messaging and Presence Protocol). XMPP servers are used worldwide
for communication, chatting and messaging over internet.\\

\noindent A chatbot (also known as a talkbot, chatterbot, Bot, IM bot, interactive agent, or Artificial Conversational
Entity) is a computer program which conducts a conversation via auditory or textual methods. Such
programs are often designed to convincingly simulate how a human would behave as a conversational
partner, thereby passing the Turing test. Chatbots are typically used in dialog systems for various practical
purposes including customer service or information acquisition. Some chatterbots use sophisticated natural
language processing systems, but many simpler systems scan for keywords within the input, then pull a
reply with the most matching keywords, or the most similar wording pattern, from a database.
\\
\noindent The app features a chatbot called Paigaam Assistant which is developed to answer the following queries
from students or faculty of the college:

\section{The Existing System}
There is no dedicated existing system for this purpose of college centric social network. Although students
have their own Facebook and WhatsApp groups for communication with their friends and get knowledge
about the activities going on in college.\\
\noindent The existing system has no chatbot to answer student or teacher queries, neither on the website nor in any
app

\section{User Requirement Analysis}
Although existing system doing their jobs as expected but in some situations, they fail.
\begin{enumerate}
\item In case of WhatsApp, some students are not comfortable in providing their phone numbers to whole
class. Reason may be anything but that they start missing out the updates that their friends provide
via these groups.
\item Inconsistency of users. This may rear but not everyone has account signed up for all services that
others used for communications. Users are distributed over a huge number and variety of
communications services available.
\item Complexity. Many applications are complex and that’s why user did not use them at their maximum
potentials.
\item Most of these services provide a limiting support for document sharing.
\item Users need to go to individual and ask them for their ID/Phone numbers to add them to the groups.
\item Most of the services takes a huge amount of data storage on user’s devices and did not provide any
backup protection.
\item Now days encryption is important but not all are able to provide encryption.
\item Chatbots have changed the way users interact with the computers.
\end{enumerate}
Users need a dedicated service that they trust, rely on and did not need to register individually to use their
features. All that needs taken into account and we are tried to provide all such facilities in this single service
and is now ready to use by the users


\section{Objective of Project }
It is a full-fledged messaging Application offering a complete, rich and innovative feature set
\begin{itemize}
\item To provide users a platform where they can simply log in using their college ID and password and
explore the college social network.
\item Users can ask frequently asked questions such as attendance, percentage etc. with the help of a
chatbot which can save their time and as well as faculty’s time.
\item Teachers can directly share assignments to students and can keep in touch with the students even
after the college hours.
\item Users can create groups, send/receive documents, images, videos
\end{itemize}